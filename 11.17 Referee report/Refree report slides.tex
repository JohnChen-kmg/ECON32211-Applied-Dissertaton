\documentclass[]{beamer}
\graphicspath{{Images/}{./}} 
\usepackage{booktabs} 
\usetheme{Madrid}
\usecolortheme{wolverine}
\usefonttheme{structurebold} % Typeset using the default sans serif font

\usepackage{palatino} % Use the Palatino font for serif text
\usepackage[default]{opensans} % Use the Open Sans font for sans serif text
\usepackage{setspace}
\usepackage{natbib}
\usepackage{multicol}
\usepackage{tikz}

\usepackage{graphicx}
\usepackage{booktabs}
\usepackage{colortbl}
\usepackage{vcell}

\useinnertheme{circles}


\title[Referee Report]{The Effect of New York City Sports Outcomes on the Stock Market\citep{p1}} 
\subtitle{Referee Report} 
\author[10763756]{Jiaxiang Chen} 
\date[November 17, 2022]{November 17, 2022} 
%----------------------------------------------------------------------------------------
\begin{document}
\begin{frame}
\titlepage 
\end{frame}

\begin{frame}
	\frametitle{The Research Question}
        \framesubtitle{\tiny{\textit{"In the lens of behavioural economics, people are irrational and their decisions are often motivated by emotions."}}}

         \begin{enumerate}
         \small
            \item \footnotesize{Stock indexes are \textbf{\alert{not wholly economically pertinent}}, the \textbf{\alert{investor's mood}} plays a part.\vskip 0.05cm
            \item \textbf{\alert{Sports events}} are important determinant of mood.\vskip 0.05cm
            \item Existing research examines the impact of local sporting outcomes on \textbf{\alert{local stock}}.}
            \textit{\tiny{For example, England national football team $\implies$ London stock exchange \citep{p2}}}
      \end{enumerate}
         
        \setbeamercolor{block title}{bg=yellow,fg=black}
	\begin{block}{Question}
        \textbf{\large{{Whether sports outcomes for New York City teams affect the national major stock indexes?}}}
        \end{block}

    \begin{columns}
        \column{.47\textwidth}
            \setbeamercolor{block title}{bg=yellow,fg=black}
            \begin{block}{\small{Assumption}}
                \scriptsize{ Many brokers and analysts are in NYC and are affected by NYC sports teams.\\
                 \hspace{10em}\textbf{$\Downarrow$}
                 \hspace{10em}Their recommendations of the stock impact many investors' decisions.}
            \end{block}
               
       \column{.47\textwidth}
            \setbeamercolor{block title}{bg=yellow,fg=black}
            \begin{block}{\small{Expectation}}
                 \scriptsize{The positive emotions of people caused by the NYC sports teams' winning.\\
                 \hspace{10em}\textbf{$\Downarrow$}
                 \hspace{10em}Theory 1: Risk taking \uparrow Return \uparrow\ Volume \uparrow\ \\ 
                 Theory 2: Risk taking \downarrow Return \downarrow\ Volume \downarrow\ }
            \end{block}
     
     \end{columns}
 \end{frame}

\section{Method}
\begin{frame}
        \frametitle{Data and Method}

        \begin{block}{Panel Data: 1949-2014 
        \tiny{(Source: Datastream, Bloomberg, sports-referemce.com)}}
         \begin{enumerate}
             \item \textbf{\small{National market index} \textit{\tiny{(Dow Jones, Nasdaq, etc)}}}
             \begin{itemize}
             \tiny
                \item Daily stock return, daily trading volume
             \end{itemize}
             
             \item \textbf{\small{8 NYC sports teams' outcome} \textit{\tiny{(Yankees, Brooklyn Nets, etc)}}}
             \begin{itemize}
             \tiny
                 \item Outcome of Playoff stage (PF), Championship final (CF)
             \end{itemize}
         \end{enumerate}
        \end{block}
                
         \begin{block}{Method 1: Ordinary Least Squares}
         \tiny{$Index_t$ : Stock return, trading volume\\
                     $Index_{t-1}$ : Yesterday price, moving average, percentage change}
             \begin{enumerate}
              \small{
                \item $Index_t=\beta_1Index_{t-1}+\beta_2TimeDummy_{it}+\sum_{i=1}^{By\ team_i}\beta_iPFoutcome_{it}$
                 \item $Index_t=\beta_1Index_{t-1}+\beta_2TimeDummy_{t}+\beta_3PFwin_{t}+\beta_4PFlose_{t}$ \scriptsize{(Pooled)}\
                \item \small $Index_t=\beta_1Index_{t-1}+\beta_2TimeDummy_{t}+\beta_3CFwin_{t}+\beta_4CFlose_{t}$ \scriptsize{(Pooled)}}
             \end{enumerate}
             \end{block}

        \begin{block}{\normalsize{Method 2: Event Study \small{(After $CFwin,\ CFlose$)}}}
        \small{
        $Cumulative\ abnormal\ return_{it}=Actual\ return_{it}-Normal\ return$
        }
        \end{block}
    
\end{frame}

\section{Result}

\begin{frame}
        \frametitle{Result}
        \framesubtitle{\textit{What do we learn?}}


        \noindent\begin{minipage}{\linewidth}
        \centering
        \resizebox{\linewidth}{!}{%
        \begin{tabular}{lcc} 
        \toprule
        \multicolumn{3}{c}{\textbf{Result ($ 10\% $ significance level)}} \\ 
        \hhline{~}
        \vcell{\textit{\textbf{Method}}} & \vcell{\textbf{OLS}} & \vcell{\textbf{Event Study}} \\[-\rowheight]
        \printcellmiddle & \printcellmiddle & \printcellbottom \\ 
        \cline{1-3}
        \noalign{\vskip\doublerulesep
         \vskip-\arrayrulewidth}
         \cline{1-3}
        \textit{\textbf{Primary Result}} & \multicolumn{1}{l}{\begin{tabular}[c]{@{}l@{}}1.Yankees PFloss $\implies$ DJIA\ volume\ 8.173\% \Uparrow\\2.No statistical significance found for others\end{tabular}} & \multicolumn{1}{l}{\begin{tabular}[c]{@{}l@{}}1.CFwin $\implies$ CAR(NYSE, S\&P500) \Uparrow~\\2.CFlose $\implies$ CAR(S\&P100, S\&P600) \Downarrow\\3.CFlose $\implies$~ Trading\ volume\Uparrow\\(DIJA, S\&P(100,500,600))\\ \end{tabular}} \\
        \bottomrule
        \end{tabular}
        }
        \end{minipage} 

 
        \footnotesize
         \item \textbf{Explanation of the difference}
                \begin{itemize}
                \setbeamerfont{item projected}{size=\tiny}
                \scriptsize
                    \item OLS:  Post-event index$\Longleftrightarrow$ Same year index when no event
                    \item Event study: Post-event index$\Longleftrightarrow$ \textbf{\alert{Near past}} index before event
                \end{itemize}


\setbeamercolor{block title}{bg=yellow,fg=black}
        \begin{block}{\small{Conclusion}}
                \setbeamerfont{item projected}{size=\tiny}
                \scriptsize
                    \item \textbf{The outcome of NYC sports team do have impact on national stock market indexes.}\\ \vskip 0.07cm
                    \tiny\textit{{Affect the emotions of analysts, and consequently their recommendation of the stock.}}
                     \begin{itemize}
                         \item \scriptsize{Good sports outcome $\Rightarrow$ Less risk averse $\Rightarrow$ Abnormal high return}\\\vskip 0.07cm
                         \tiny\textit{{Positive emotions yielded a decreased estimation of risk probability. \citep{p3}}}\\
                         \item \scriptsize{Bad sports outcome $\Rightarrow$ Less risk averse $\Rightarrow$ High volume}\\\vskip 0.07cm
                         \tiny\textit{{Negative emotions such as embarrassment and anger increased risk taking. They want to select high-risk, high-reward choices to alter their state of mind. \citep{p4}}}
                         
                     \end{itemize}
        \end{block}
        
\end{frame}

\section{Shortcomings and Possible Improvements}

\begin{frame}
        \frametitle{Shortcomings and Possible Improvements }
        \framesubtitle{Research Design}

        \begin{columns}
    \column{.45\textwidth}
        \setbeamercolor{block title}{bg=yellow,fg=black}
        \begin{block}{\small{Problems}}
            \small \textbf{\alert{1.} The topic is not contributive }\\\vskip 0.01cm
                \scriptsize
                \begin{itemize}
                \item Interesting at first glimpse, but casual relation is obvious.\\
                \tiny{\textit{Sports event $\rightarrow$ emotions $\rightarrow$ invest decisions}} 
                    \scriptsize
               \item Mostly for curiosity? Policy  or applications? 
                \end{itemize}
            
            \small \textbf{\alert{2.} The assumption is vague}\\
                \tiny
                \begin{itemize}
                \textit{"NYC sports$\rightarrow$ analysts emotions"}
                \end{itemize}
                  \vskip -0.25cm  
                \begin{itemize}
                \scriptsize 
                \item Literature review: independent variable always has a clear effect on investors:\\\vskip 0.15cm 
                \tiny{\textit{Daylight and weahter $\rightarrow$ Investors' emotions \citep{p5}\citep{p6}}}
                \vskip -0.15cm
            \scriptsize
                  \item What percent are based in NY?\\
                        What percent are sports fans? 

                \end{itemize}
                
        \end{block}

 
    \column{.025\textwidth}
    
      \vskip 0.7cm
      \big\Rightarrow
           
     \column{.45\textwidth}
        \setbeamercolor{block title}{bg=yellow,fg=black}
        \begin{block}{\small{Possible improvements}}
            \small \textbf{\alert{1.} Think of applications}\\\vskip 0.01cm
                \scriptsize
                \begin{itemize}
                \item For example, after major event, analysts with a season ticket should be given trivial tasks.
                \item Maybe unpublishable? But it has published.
                \end{itemize}
   
            \small \textbf{\alert{2.} Conduct surveys to circuit it}\\
                \tiny
                \begin{itemize}
                \scriptsize 
                \item Target the analysts who are indeed affected by the sports events.
                \item Also, we can obtain the list of their recommended stocks.\\~\\
                \vskip 0.6cm
                 \textit{}
                \end{itemize}
        \end{block}
 
 \end{columns}

 
 \end{frame}



\begin{frame}
        \frametitle{Shortcomings and Possible Improvements }
        \framesubtitle{Econometrics}
        \begin{columns}
        \column{.45\textwidth}
            \begin{block}{\small{Problems}}
                \small \textbf{\alert{1.}Heterogeneity \tiny{(OLS, Event study)}} \vskip -0.1cm 
                    \begin{itemize}
                     \scriptsize
                    \item Different teams shouldn't be expected to have the same impact over time.\\ \vskip 0.03cm
                    \tiny{\textit{1.Yankees's impact is different from Mets'\\ \vskip 0.03cm
                                  2.Brooklyn nets' impact now is different from 50s}}
                    \end{itemize}
                    
                \small \textbf{\alert{2.} Multicollinearity \tiny{(OLS, Event study)}}\vskip -0.1cm 
                    \begin{itemize}
                     \scriptsize
                    \item Different teams can play against each other.\\ \vskip 0.03cm
                    \tiny{\textit{Yankees's VS Mets in the PF or CF}}
                    \end{itemize}

                \small \textbf{\alert{3}. OVB \tiny{(Event study: near past)}} \vskip -0.1cm 
                    \begin{itemize}
                     \scriptsize
                    \item Championship stage happen the same time every year.     \vskip -0.3cm  
                    \item The seasonality of economy impact stock market index.
                    \end{itemize}

            \end{block}

 
    \column{.025\textwidth}
      \vskip 0.5cm
      \big\Rightarrow

      
   \column{.45\textwidth}
        \begin{block}{\small{Possible Improvements}}
        
         \small \textbf{\alert{1.} Fixed effect estimators}\vskip -0.2cm 
                \begin{itemize}
                   \textbf{\tiny{(OLS, Event study)}} 
                \end{itemize} \vskip -0.3cm 
                    \begin{itemize}
                     \scriptsize
                    \item Fixed effects estimators control the individual effects and time effects. \vskip 0.3cm
                    \end{itemize}

         \small \textbf{\alert{2.} Drop some observations}\vskip -0.2cm
                \begin{itemize}
                   \textbf{\tiny{(OLS, Event study)}} 
                \end{itemize} \vskip -0.3cm
                    \begin{itemize}
                     \scriptsize
                    \item Troublesome observations won't be a lot, so we can drop them.\\ \vskip 0.3cm
                    \end{itemize}

            \small \textbf{\alert{3}. Add control variable} \vskip -0.2cm
                \begin{itemize}
                   \textbf{\tiny{(Event study: near past)}} 
                \end{itemize}\vskip -0.3cm
                    \begin{itemize}
                     \scriptsize 
                    \item Use time variable to control the seasonality.\\ \vskip 0.03cm
                    \tiny{\textit{(If it is proper for event study)}}
                    \end{itemize}
                    \vskip 0.1cm
                    \begin{itemize}
                        \textbf{}
                    \end{itemize}
                  

        \end{block}

  \end{columns}
\end{frame}




\begin{frame} 
	\frametitle{References}
	
	\begin{thebibliography}{99}
		\scriptsize % Reduce the font size in the bibliography

        	\bibitem[Ashton et al, 2003]{p2}
			Ashton, John, Bill Gerrard, and Robert Hudson, (2003)
			\newblock Economic impact of national sporting success: Evidence from the London Stock Exchange
			\newblock \emph{Applied Economic Letters} 10, 783 – 785.

   		  \bibitem[Cao et al, 2005]{p6}
			Cao, Melanie, and Jason Wei, (2005)
			\newblock Stock market returns: A note on temperature anomaly
			\newblock \emph{Journal of Banking and Finance} 29, 1559 - 1573.

                \bibitem[Hirshleifer et al, 2003]{p5}
			Hirshleifer, David, and Tyler Shumway, (2003)
			\newblock Good day sunshine: Stock returns and the weather
			\newblock \emph{Journal of Finance} 58, 1009 - 1032.

                \bibitem[Johnson et al, 1983]{p3}
			Johnson, Eric J., and Amos Tversky, (1983)
			\newblock Affect, Generalization, And The Perception Of Risk
			\newblock \emph{Journal Of Personality And Social Psychology} 45, 20 – 31.

                \bibitem[Leith et al, 1996]{p4}
			Leith, Karen Pezza, and Roy F. Baumeister, (1996)
			\newblock Why Do Bad Moods Increase Self-Defeating Behavior? Emotion, Risk Tasking, And Self-Regulation
			\newblock \emph{Journal Of Personality And Social Psychology} 71, 1250 – 1267.

   
  	      \bibitem[Levy, 2015]{p1}
			Levy, Nir, (2015)
			\newblock The Effect of New York City Sports Outcomes on the Stock Market
			\newblock \emph{Undergraduate Economic Review} 12.1, 8.

   
	\end{thebibliography}
\end{frame}
\end{document} 
