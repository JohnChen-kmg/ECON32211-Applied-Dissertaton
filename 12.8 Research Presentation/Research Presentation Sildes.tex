\documentclass[]{beamer}
\graphicspath{{Images/}{./}} 
\usepackage{booktabs} 
\usetheme{Madrid}
\usecolortheme{beaver}
\usefonttheme{structurebold} % Typeset using the default sans serif font
\definecolor{brown(web)}{rgb}{0.65, 0.16, 0.16}
\definecolor{lightred}{rgb}{0.34, 0.26, 0.25}
\definecolor{green}{rgb}{0.22, 0.39, 0}
\definecolor{foo}{rgb}{0.2,0.2,0.7}

\usepackage{palatino} % Use the Palatino font for serif text
\usepackage[default]{opensans} % Use the Open Sans font for sans serif text
\usepackage{setspace}
\usepackage{natbib}
\usepackage{multicol}
\usepackage{tikz}

\usepackage{graphicx}
\usepackage{booktabs}
\usepackage{colortbl}
\usepackage{vcell}

\useinnertheme{circles}

\newcommand<>{\ColorAlert}[2][foo]{\begingroup%
\setbeamercolor{alerted text}{fg=#1}\alert{#2}\endgroup}

\title[Research Proposal Presentation]{The Impact of the State Media on the Destination of Internal Migrants in China} 
\subtitle{Research Proposal Presentation} 
\author[10763756]{Jiaxiang Chen} 
\date[December 8, 2022]{December 8, 2022} 
%----------------------------------------------------------------------------------------
\begin{document}
\begin{frame}
\titlepage 
\end{frame}

\section{Motivation}
\begin{frame}
    \frametitle{Motivation}
    \framesubtitle{\textit{\tiny{According to the International Labour Organization, Internal migration in China is one of the most extensive in the world.}}}
    
    \begin{enumerate}
    \item \alert{History Context}
    \vskip 0.2cm
    \begin{itemize}
    \footnotesize{
        \item 
        Since 1979, China's economic reform has made manufacturing thriving in its eastern part.
        \vskip 0.1cm
        \item Policies that strictly prevented internal migration since 50s was partially discarded.
        \vskip 0.1cm
        \item  The rural labor began to migrate to search for working opportunities.
        }
    \end{itemize}
    \vskip 0.3cm

    \item \alert{Understanding Migration Decisions is Important}
    \vskip 0.2cm
    \begin{itemize}\footnotesize{
        \item Help us to understand more about the migration pattern and labor market situations \citep{p1}.
        \vskip 0.1cm
        \item Relevant polices designing to control the migration and to decrease the labor market frictions.}
        \\~\\
        \\~\\
    \end{itemize}
    \end{enumerate}
    
\end{frame}


\section{Information Model}
\begin{frame}
    \frametitle{Information and Migrants' Decisions}
    
    \setbeamercolor{block title}{bg=brown(web)}
	\begin{block}{Canonical Migration Choice Model \citep{p2}}
        \begin{itemize}
        \scriptsize{
        \item 
        A person will move from $o$ to $d$, if utility of expected income of moving exceed that of staying plus the moving cost,}
            \begin{equation*}
            {Move_{od} = \begin{cases}
                                1\  \  if\ U(E(Income_{move})) \geqslant U(Income_{stay}) + U(Cost_{move})\\
                                0\   \  else
                               \end{cases}}
            \end{equation*}
         \item
         \scriptsize{
         Individual face \textbf{\alert{incomplete information}} and \textbf{\alert{uncertainty}} about the potential destination. For simplicity, expected income of moving is assumed to has a normal distribution: $E(Income_{move}) \sim  N(\mu,\ \sigma^2)$. }
         \end{itemize}
         
        \end{block}

    \setbeamercolor{block title}{bg=brown(web)}
	\begin{block}{Information Update the Belief}
         \scriptsize{Assuming a risk averse potential migrant has \textbf{\alert{more information}},}
         
        \begin{itemize}
        \item
         \scriptsize{
         \textbf{\alert{Perceived mean of income}}: Positive info increase $\mu$, negative info decrease $\mu$. }
         
         \item
         \scriptsize{
         \textbf{\alert{Perceived certainty}}: All types of info decrease $\sigma^2$, making the migration less risky.}

           \item
         \scriptsize{
         Positive information increase chance of moving, negative one's effect is ambiguous.}

         
         \end{itemize}
         \vskip 0.08cm
         \scriptsize{
         With \textbf{\alert{updated belief}} $E(Income^*_{move}|Info) \sim N(\mu*,\ \sigma*^2)$, one may \textbf{\alert{alter the decision}},}
        \end{block}      
  \end{frame}


\section{LR I}
\begin{frame}
    \frametitle{Literature Review I: }
    \framesubtitle{\scriptsize{\textit{Information is Important in Making Migration Decisions}}}
        \vskip 0.1cm
        
    \begin{enumerate}
        \item  \alert{\small{Most works in the area focus on the role of networks or linguistic and cultural enclaves.}}
        \vskip 0.2cm
        
        \begin{itemize}

            \item \footnotesize{Using data of rural China, \cite{p3} found that more experienced migrants a village has, more likely other villagers will migrate.}
            \vskip 0.2cm

            \item \footnotesize{\cite{p5} shows that an immigrant in US  with a larger network, is more likely to be employed and to have a higher income.}
            \vskip 0.2cm

        \end{itemize}
        \vskip 0.2cm

     \item \alert{\small{However, information can be in other forms. For example, news on TV or newspaper.}}
     \vskip 0.2cm

     \begin{itemize}
         \item \footnotesize{Access to more TV stations in Indonesia reduced the likelihood of moving\citep{p6}.}\\
         \vskip 0.1cm
         \tiny{\textit{(Plausibly by correcting optimistic expectations about migration)}}
         \vskip 0.2cm

         \item \footnotesize{Specifically, the content of news should have an impact.}
         
     \end{itemize}
    \end{enumerate}

    \end{frame}


\section{LR II}
\begin{frame}
    \frametitle{Literature Review II: }
    \framesubtitle{\scriptsize{\textit{Moving to jobs: The role of information in migration decisions \citep{p7}}}}
        \vskip 0.2cm

        \begin{enumerate}
            \item \small{\alert{Question: Do exposure to news about fracking affect migration?}}
            \vskip 0.1cm
            
            \begin{itemize}\scriptsize{
                \item In US, fracking booms lead to increase in local employment and earnings.
                \vskip 0.08cm
                
                \item Newspaper and TV began touting its benefits or adverse side effects.}
            \end{itemize}
            \vskip 0.15cm

            \item \small{\alert{Methodology}}
             \vskip 0.1cm
            \begin{itemize} 
             \footnotesize{\item} \begin{center}
                \footnotesize{
                $ Migration_{o\Rightarrow s}=\beta_1 NewsExposure_{s\ fracking}+\beta_2 Control_{os}+\beta_{3,4} FixedEffect_{o,s}$}
                \end{center}\end{itemize}
                \vskip 0.15cm

            \item \small{\alert{Conclusion}}
             \vskip 0.1cm
            \begin{itemize}\scriptsize{
                \item One news article about fracking in state s increase migration flow to s by 2.4\%.
                \vskip 0.08cm
                \item Migration flows are less responsive to \alert{negative news}, but its effect is \alert{positive}}.\\
            \end{itemize}
        \end{enumerate}
        \vskip -0.2cm


    \begin{center}
    \vskip -0.2cm
    \begin{column}{0.95\textwidth}
    \vskip -0.2cm
        \setbeamercolor{block title}{bg=green}
	\begin{block}{Explanation and Intuition}
        \begin{itemize}
        \footnotesize{ \item The negative effect can have positive impact.\\
        \tiny{Reduction in uncertainty $>$ Reduction in perceived mean of income}
        
        \vskip 0.12cm
        
        \item
        \footnotesize{People have a prior belief of the news\scriptsize{(positive about fracking)}, so their perception are the real base for updating belief:} }\\ \tiny{\textit{$E(Income^*_{move}|Perception\ of\ News)$ instead $E(Income^*_{move}|\ News)$  }}
        \vskip 0.12cm

        \item
        \footnotesize{Overall, news content can impact the migrant's decision of destination.}

        \end{itemize}
       \end{block}
    \end{column}
    \end{center}


\end{frame}


\section{China State Media}
\begin{frame}
    \frametitle{China's State Media}
    \framesubtitle{\scriptsize{\textit{In the context of China, there can be influential news sources impact people's migration decision}}}

    
    \begin{itemize}
  
         \footnotesize{\item  State media who monopolized the market are very authoritative information sources. For example,}

                   \begin{center}         
     \begin{columns}
        
            \column{.45\textwidth} 
            \begin{block}{\small{1. News Simulcast}}

            \begin{itemize}
            \scriptsize{
                \item Most watched news program in China since 1978.
                \item Covering major domestic and foreign affairs.}
            \end{itemize}

            \end{block}

            
            \column{.45\textwidth}
            \begin{block}{\small{2. People's Daily }}

            \begin{itemize}
            \scriptsize{
                \item Official newspaper of the government since 1948.
                \item The government provides its policy induction and viewpoint directly. }
            \end{itemize}
            \end{block}
   
     \end{columns}
      \end{center}
     \vskip 0.2cm
    

            \footnotesize{\item However, the coverage of the event can be culturally, ideologically, and politically situated\citep{p8}.}\\
            \vskip 0.04cm

            \begin{enumerate}
            \setbeamerfont{item projected}{size=\scriptsize}

            \item\scriptsize{Reported foreign affairs are often negative\citep{p9}, which \alert{manipulate the viewer's  viewpoint} about foreign countries\citep{p10}.
            \vskip 0.05cm
            
            \item The report of the domestic affairs are often positive, always showing the positive prospects across the nation. }
            \end{enumerate}

            \vskip 0.05cm

            \ColorAlert{\footnotesize{Since the state media can render positive economy prospects in reported places(though might be biased), could they affect migrant's decisions? }}
            
           \end{itemize}

\end{frame}



\section{Research Question}
\begin{frame}
    \frametitle{Research Question:}
    \framesubtitle{\normalize{\textbf{Does the State Media Impact the Destinations of Internal Migrants in China?}}}
    \vskip 0.05cm

    \begin{itemize}
        \item \alert{Hardly any Similar Research}\\
        \vskip 0.1cm
        \footnotesize{In the existing literature regarding the China's internal migration, the role of exposure to news hasn't been scrutinized.}
        \vskip 0.15cm

        \normalsize{\item \alert{Assumption}}\\
        \vskip 0.1cm
        \footnotesize{Most state media's domestic news are positive, the exposure of one province update the viewers' perception of that province's economy opportunities, modifying the chance of moving there.}
        \vskip 0.12cm

    \end{itemize}
        
    \setbeamercolor{block title}{bg=brown(web)}
    \begin{block}{Expectation}
    

        \footnotesize{Control for other factors, we expect the exposure of a particular province $s$,}
        \vskip 0.08cm
        
        \begin{enumerate}
            \footnotesize{\item Increase the chance a migrant select $s$ as destination.}\\
            \vskip 0.06cm
            \tiny{\textit{The coverage convert into more positive perception.}}
            \vskip 0.1cm
            
            \footnotesize{\item Decrease the chance a migrant select $s$ as destination.}\\
            \vskip 0.06cm
            \tiny{\textit{The coverage convert into more negative perception. People have prior belief about province $s$}}
            \vskip 0.1cm
            
            \footnotesize{\item No impact on the migrant's decision making.}\\
            \vskip 0.06cm
            \tiny{\textit{People are more relied on other types of information.}}
            
        \end{enumerate}
        
    \end{block}

\end{frame}


\section{Possible Method}
\begin{frame}
    \frametitle{Possible Method}

    \setbeamercolor{block title}{bg=brown(web)}
    \begin{block}{Ordinary Least Squares}
    \vspace{0.1cm}

      $ Migration_{ost}=\beta_1 NewsExposure_{st}+\beta_2 Control_{os}+\beta_{3,4} FixedEffect_{ot,st}$

     \vskip 0.12cm

      \setlength{\leftmargini}{20pt}
      \begin{itemize}
         \footnotesize{
          \item \alert{$Migration_{ost}$}: Percentage of migrants move from province $o$ to $s$ in the total migrants moving out $o$ in year $t$.
             \vskip 0.1cm 
             \item \alert{$NewsExposure_{st}$}: Percentage of province $s$ related news on News Simulcast or People's Daily.
             \vskip 0.1cm
             \item \alert{$Control_{os}$}: Control factors between province $o$ and $s$ (Distance).
             \vskip 0.1cm
             \item \alert{$FixedEffect_{ot,st}$}: Province fixed effects in year $t$ (Income, Unemployment).
             }
             \end{itemize}
             
    \end{block}

    \setlength{\leftmargini}{12pt}
    \begin{itemize}
        \item \alert{Additional Transformation}
        \vskip 0.15cm
        
       \setlength{\leftmarginii}{10pt}
       \setbeamerfont{itemize/enumerate body}{size=\footnotesize}
        \begin{enumerate} \footnotesize{
            \item Add previous years' $Migration_{os,t-n}$ to account for the network effect.
            \vskip 0.1cm

            \item Include the previous years' $NewsExposure_{s,t-n}$ to capture the potential lag effect.
             \vskip 0.1cm

            \item Add square term of $NewsExposure_{st}$ to seize the decreasing returns to information\citep{p7}.
            }
        \end{enumerate}
        \end{itemize}
    
\end{frame}


\section{Data Sources and Potential Problems}
\begin{frame}
    \frametitle{Data Source and Potential Problems}

    \begin{itemize}
        \item \alert{Data Source}
        \vskip 0.1cm
        
        \setbeamerfont{itemize/enumerate body}{size=\footnotesize}
        \begin{enumerate}\footnotesize{
            \item \alert{Migration Data}: Household Surveys, Panel Data Set of Previous Research
            \vskip 0.1cm
            
            \item \alert{News Data}: Database of News Simulcast and People's daily are available, but to extract the exposure of province need data mining. Luckily, the code for similar process is online(Kaggle).
            }
        \end{enumerate}
        \vskip 0.25 cm

        \item \alert{Potential Problems}
        
        \setbeamerfont{itemize/enumerate body}{size=\footnotesize}
        \vskip 0.1cm
         \begin{enumerate}\footnotesize{
            \item \alert{Data Matching}: Data from different sources may not be in the same time period.
            \vskip 0.1cm
            
            \item \alert{News Exposure Rate}: News exposure rate should varied across the province but data haven't been found.\\
            \textit{\tiny{(May use possession rate of TV or literacy rate instead.)}}
            \vskip 0.1cm

            \item \alert{Types of News}: We cannot expect different types of news to have same effect (Economy, Politics)\\
            \textit{\tiny{(More advanced text mining technique may required.)}}
            }
        \end{enumerate}
    \end{itemize}\\~\\
    
\end{frame}



\section{Reference List}
\begin{frame} 
	\frametitle{Reference List}
	
	\begin{thebibliography}{99}
		\tiny % Reduce the font size in the bibliography

   		  \bibitem[Fang et al, 2009]{p1}
			Fang, C, Du, Y, Wang, M. (2009)
			\newblock Migration and labor mobility in China.
			\newblock \emph{} 

                \bibitem[Farré, 2013]{p6}
			Farré, L., Fasani, F. (2013)
			\newblock Media exposure and internal migration—Evidence from Indonesia.
			\newblock \emph{Journal of Development Economics} 102, 48 - 61.

                \bibitem[Munshi, 2003]{p5}
			Munshi, K. (2003)
			\newblock  Networks in the modern economy: Mexican migrants in the US labor market. 
			\newblock \emph{The Quarterly Journal of Economics} 118(2), 549 – 599.

                \bibitem[Pan, 2020]{p10}
			Pan,J., Shao,Z., XU,Y. (2020)
			\newblockThe Effects of Television News Propaganda: Experimental Evidence from China. 
			\newblock \emph{Available at SSRN 3579148}

              \bibitem[Sjaastad, 1962]{p2}
			Sjaastad, L.A (1962)
			\newblock The costs and returns of human migration, 
			\newblock \emph{Journal of political Economy} 70.5, 80 – 93.

                \bibitem[Wilson, 2021]{p7}
			Wilson,R. (2021)
			\newblock Moving to jobs: The role of information in migration decisions. 
			\newblock \emph{Journal of Labor Economics} 39(4), 1083 - 1128.

                \bibitem[Wang, 1993]{p8}
			Wang,S. (1993)
			\newblock The New York Times’ and Renmin Ribao’s news coverage of the 1991 Soviel coup: A case study of international news discourse. 
			\newblock \emph{Text-Interdisciplinary Journal for the Study of Discourse} 13(4), 559 - 598.

               \bibitem[Xu, 2019]{p9}
			Xu, Z., Clark,A.M. (2019)
			\newblockIt is a Scary and Dangerous World: A Content Analysis of Xinwen Lianbo’s News Coverage of Foreign Affairs.
			\newblock \emph{}
   
                \bibitem[Zhao, 2003]{p3}
			Zhao, Y. (2003)
			\newblock The role of migrant networks in labor migration: The case of China, 
			\newblock \emph{Contemporary Economic Policy} 21(4), 500 – 511.
   

	\end{thebibliography}
\end{frame}
\end{document} 
